\documentclass{beamer}
\usepackage[utf8]{inputenc}

% Math and CS packages
\usepackage{amsmath, amsthm, amssymb, amsfonts}
\usepackage{hyperref}
\hypersetup{
    colorlinks=true,
    linkcolor=blue,
    filecolor=magenta,      
    urlcolor=cyan,
    pdftitle={Equivalence Relations on Algebraic Cycles},
    pdfpagemode=FullScreen,
}
\urlstyle{same}

\title{Equivalence Relations on Algebraic Cycles}
\author{Nathan Stefanik}


\begin{document}

%------------------------------------------------------------
\frame{\titlepage}
%------------------------------------------------------------

%------------------------------------------------------------
\begin{frame}{History and Motivation}
  \pause
  \begin{itemize}
    \item $19^{\text{th}}$ century - significant work on functions on algebraic curves \pause
    \item Riemann-Roch Theorem, Abel-Jacobi Theorem major contributions using divisors (algebraic cycles of
      codimension 1) \pause
    \item In higher dimensions, behavior of varieties becomes more complicated and these theorems don't
      nicely extend
  \end{itemize}
\end{frame}
%------------------------------------------------------------

%------------------------------------------------------------
\begin{frame}{Algebraic Cycles}
  \pause
  Let $X$ be any variety defined over a field $k$.\pause
  \begin{definition}
    A \textbf{cycle $Z$ of codimension $r$} on $X$ is an element of the free abelian group generated
    by the closed irreducible subvarieties of codimension $r$ of $X$. It is a finite formal sum $\sum n_i [V_i]$ where
    $n_i$ are integers and $V_i$ are subvarieties. 
  \end{definition}\pause
  We denote by $C^r(X)$ the group of all cycles of codimension $r$ on $X$.
\end{frame}
%-----------------------------------------------------------

%-----------------------------------------------------------
\begin{frame}{Algebraic Cycles}
  \begin{itemize}
    \item $C^r(X)$ is naturally quite large \pause
    \item Use equivalence relations to develop more intuition on the geometry of $X$.
  \end{itemize}
\end{frame}
%-----------------------------------------------------------

%-----------------------------------------------------------
\begin{frame}{A Nice Example}
  Let $r=1$. A cycle of codimension one is a divisor. 
  \begin{definition}
    Two divisors $D_1$ and $D_2$ on $X$ are linearly equivalent if there exists a rational function on $X$ such that
    $D_1 - D_2 = (f)_0 - (f)_\infty$, where $(f)_0$ denotes the divisor of zeros and $(f)_\infty$ denotes the
    divisor of poles. 
  \end{definition}\pause

  \begin{example}
    Let $C^1_{\text{lin}}(X)$ denote the group of divisors linearly equivalent to 0. Then, the quotient group 
    $C^1(X) / C^1_{\text{lin}}(X)$ is $\text{Pic } X$, the group of linear equivalence classes of divisors on $X$.
  \end{example}
\end{frame}
%-----------------------------------------------------------

%-----------------------------------------------------------
\begin{frame}{A Nice Example}
  For $X = \mathbb{P}^n$, $C^1(X) / C^1_{\text{lin}}(X) \cong \mathbb{Z}$.
\end{frame}
%-----------------------------------------------------------

%-----------------------------------------------------------
\begin{frame}{Rational Equivalence}
  \begin{itemize}
    \item Same as linear equivalence in codim 1 \pause
  \end{itemize}
  \begin{definition}
    Two cycles $Z_1$ and $Z_2$ of codimension $r$ on $X$ are \textbf{rationally equivalent} if there is a cycle
    $Z$ on $X \times \mathbb{P}^1$, which intersects each fiber $X \times \{t\}$ in something of codimension $r$, and
    such that $Z_1$ and $Z_2$ are obtained respectively by intersecting $Z$ with the fibers $X\times\{0\}$ 
    and $X\times\{1\}$.
  \end{definition}
\end{frame}
%-----------------------------------------------------------

%-----------------------------------------------------------
\begin{frame}{Rational Equivalence is an Equivalence Relation}
  \begin{itemize}
    \item Reflectivity: $Z_1 \sim_{rat} Z_1$. Take $Z = Z_1 \times \mathbb{P}^1$ \pause
    \item Symmetry: $Z_1 \sim_{rat} Z_2 \implies Z_2 \sim_{rat} Z_1$. Apply automorphism of $\mathbb{P}^1$ that 
      interchanges $0$ and $1$. \pause
    \item Transivity: $Z_1 \sim_{rat} Z_2, Z_2 \sim_{rat} Z_3 \implies Z_1 \sim_{rat} Z_3$. Let $Z \subseteq 
      X \times \mathbb{P}^1$ give $Z_1 \sim_{rat} Z_2$ and $Z' \subseteq X \times \mathbb{P}^1$ give $Z_2 \sim_{rat}
      Z_3$. Then, $Z+Z'-Z_2 \times \mathbb{P}^1$ gives $Z_1 \sim_{rat} Z_3$. 
  \end{itemize}
\end{frame}
%-----------------------------------------------------------

\newtheorem*{remark}{Remark}
%-----------------------------------------------------------
\begin{frame}{Algebraic Equivalence}
   \begin{itemize}
    \item Use same construction for rational equivalence \pause
  \end{itemize}
  \begin{definition}
    Let $C$ be an irreducible curve, and $a, b \in C$ be any two points. 
    Two cycles $Z_1$ and $Z_2$ of codimension $r$ on $X$ are \textbf{algebraically equivalent} if there is a cycle
    $Z$ on $X \times C$, which intersects each fiber $X \times \{t\}$ in something of codimension $r$, and
    such that $Z_1$ and $Z_2$ are obtained respectively by intersecting $Z$ with the fibers $X\times\{a\}$ 
    and $X\times\{b\}$.
  \end{definition}\pause
  \begin{remark}
    $C^r_{rat}(X) \subset C^r_{alg}(X) \subset C^r(X)$
  \end{remark}
  
\end{frame}
%-----------------------------------------------------------

%-----------------------------------------------------------
\begin{frame}{Other Adequate Equivalence Relations}
  \begin{itemize}
    \item Numerical equivalence, torsion equivalence, homological equivalence \pause
    \item $C^r \supseteq C^r_{num} \supseteq C^r_{hom} \supseteq C^r_{\tau} \supseteq C^r_{alg}$
  \end{itemize}
\end{frame}
%-----------------------------------------------------------

%-----------------------------------------------------------
\begin{frame}{Further Study and Applications}
  \begin{itemize}
    \item intermediate jacobians, k-theoretic and cohomology methods, relative cycles \pause
    \item behavior of algebraic cycles useful in intersection theory, algebraic k-theory, and hodge conjecture
  \end{itemize}
\end{frame}
%-----------------------------------------------------------

%-----------------------------------------------------------
\begin{frame}{References}
  \begin{itemize}
    \item Spencer Bloch (1980) "Lectures on Algebraic Cycles", Mathematics Department Duke University.
    \item Robin Hartshorne (1974) "Equivalence Relations on Algebraic Cycles and Subvarieties of Small Codimension", 
      AMS.
  \end{itemize}
\end{frame}
%-----------------------------------------------------------



\end{document}

